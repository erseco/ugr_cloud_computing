\chapter{Introducción}

\subsubsection{Para	 el	 diseño	 del	 sistema	 de	 almacenamiento	 de	 alta	 disponibilidad	 en	 Cloud,	 hay	 que	
considerar	los	componentes	siguientes	que	deben	ser	desplegados en	la	arquitectura	que	
se	ha	provisto	dentro	de	OpenStack:}

\begin{enumerate}
\item
Servicio	 de	 alta	 disponibilidad / balanceador	 de	 carga.	 Este	 servicio	 será	 el	
encargado	 de	 enrutar y	 balancear	 las	 peticiones	 de	 los	 clientes	 a	 los	 servicios	 de	
almacenamiento	redundantes	que	se	desplegarán.	 Se	usará	HAproxy o	NGINX.	Se	
recomienda	utilizar	NGINX,	debido	a	su	facilidad	de	uso.
\item
Servicio	 de	 almacenamiento.	 Es	 el	 servicio	 de	 almacenamiento	 y	 gestión	 de	 los	
datos	en	 cloud.	Este	 servicio	 se	encontrará	 replicado	en	al	menos	dos	nodos	de	la	
infraestructura	y	dará	soporte	de	almacenamiento	en	cloud.	Se	usará	la	plataforma	
de	almacenamiento	de	ficheros	owncloud.

\item
Servicio	 de	 auto-escalado	 dinámico.	 Este	 servicio	 permite	 que	 se	 creen	 o	
destruyan	 nuevos	 contenedores	 para	 el	 servicio	 de	 almacenamiento.	 El	 servicio	
debe	ser	programado	en	el	lenguaje	que	prefieras.

\item
Servicios	de	bases	de	datos.	El	servicio	de	almacenamiento	requiere	de	un	sistema	
de	 base	 de	 datos	 para	 poder	 almacenar	 toda	la	información	 de	 usuarios,	 ficheros,	
metadatos,	 etc.	 Para	 ello	 como	 mínimo	 se	 usará	 un	 nodo,	 con	
MariaBD/MySQL/PostgreSQL.

\item
Servicio	 de	 autenticación.	 Este	 servicio	 se	 encargará	 de	 habilitar	 el	 acceso	 a	los	
usuarios	 de	 owncloud en	 el	 sistema	 de	 almacenamiento	 en	 cloud.	 El	 servicio	 de	
directorio	 de	 usuarios	 será	 LDAP y	 podrá	 estar	 compartido	 dentro	 del	 nodo	 de	
Base	de	Datos.

\end{enumerate}